\documentclass[11pt]{article}
\usepackage{fullpage,hyperref}\setlength{\parskip}{3mm}\setlength{\parindent}{0mm}
\begin{document}

\begin{center}{\bf
Homework 7. Due by 11:59pm on Sunday 10/27.

Negligence, mistakes \& how to avoid them

}\end{center}


In past years, the class on correcting errors has involved looking at statistics journals and noticing the small number of published corrections. Perhaps statisticians are so rigorous, and the peer review so careful, that errors are rare. This year, I am presenting from a different perspective. You are asked to respond to the following letter, imagining that you are an editor of JRSSB. You are asked to write one paragraph outlining how you would handle the situation.

I have sent the same letter to the actual editors of JRSSB, and their response will be revealed in class. Remember that JRSSB editors are busy, so you should read the letter, and optionally the arXiv post that it refers to, but you may not have time to read the original paper under scrutiny. You should read pages 12--14 of {\em On Being a Scientist} as background.

Submit your paragraph as a pdf via Canvas. Unlike previous homeworks, you do not have to edit this tex file, though you are welcome to do so.

{\fontfamily{cmtt}\selectfont

Dear Professors Doss, Witten and Yao,

This email is to remind you that you have not yet responded to my request that you consider the arXiv post at https://arxiv.org/abs/2409.12173 and act to correct the scientific record in JRSSB if you find that the claim in this post is correct. As I have previously pointed out, the error is not a subtle epidemiological question of how to deal with under-reporting (as the authors claim). It is simply a basic miscalculation in the numerical evaluation of the capabilities of the new methodology proposed in the paper. This error makes their methodology look very strong compared to alternatives on their example, when this is not actually the case. 

At this point, there are two distinct issues. First, there is the scientific issue discussed in the arXiv post. Second, there is the procedural issue concerning the apparent difficulty in obtaining a reasonable resolution following due process once an error has been identified. The only action you offered was to send a manuscript on this topic for peer review if I submit one to JRSSB, while reminding me that, "a paper that merely corrects some errors without methodology innovation is unlikely to be accepted by Series B, but can be published in an online forum, such as arXiv." This is a policy to consider a correction carefully and then refuse to publish it, which is not a fair process. Please let me know if that is not what you meant.

On October 28, I am teaching a federally mandated PhD seminar class on handling research errors for responsible conduct in research and scholarship (https://ionides.github.io/810f24/). I will be using this situation as a case study. This email and any response from the JRSSB editors (or the lack of response) will be relayed to the students. Hopefully this transparent process will be instructive for the PhD students, whatever conclusion it may arrive at.

Sincerely,
Ed Ionides

}
\end{document}

